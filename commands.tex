% در این فایل، دستورها و تنظیمات مورد نیاز، آورده شده است.
%-------------------------------------------------------------------------------------------------------------------

% در ورژن جدید زی‌پرشین برای تایپ متن‌های ریاضی، این سه بسته، حتماً باید فراخوانی شود
\usepackage{amsthm,amssymb,amsmath}
% بسته‌ای برای تنطیم حاشیه‌های بالا، پایین، چپ و راست صفحه
\usepackage[top=25mm, bottom=25mm, left=22mm, right=22mm]{geometry}
% بسته‌‌ای برای ظاهر شدن شکل‌ها و تصاویر متن
\usepackage{graphicx,xcolor}
% بسته‌ای برای رسم کادر
\usepackage{framed} 
% بسته‌‌ای برای چاپ شدن خودکار تعداد صفحات در صفحه «معرفی پایان‌نامه»
\usepackage{lastpage}
% بسته‌ و دستوراتی برای ایجاد لینک‌های رنگی با امکان جهش
\usepackage[linktocpage=true,colorlinks,citecolor=blue,linkcolor=blue]{hyperref}
% چنانچه قصد پرینت گرفتن نوشته خود را دارید، خط بالا را غیرفعال و  از دستور زیر استفاده کنید چون در صورت استفاده از دستور زیر‌‌، 
% لینک‌ها به رنگ سیاه ظاهر خواهند شد که برای پرینت گرفتن، مناسب‌تر است
%\usepackage[pagebackref=false]{hyperref}
% بسته‌ لازم برای تنظیم سربرگ‌ها
\usepackage{fancyhdr}
\usepackage{setspace}
\usepackage{alltt}
\usepackage{zref-perpage}
\usepackage{algorithm}
\usepackage{algorithmic}
\usepackage{subfigure}
\usepackage{natbib}
%\usepackage{bibunits}
\usepackage{bibentry}
\usepackage{titlesec}
\usepackage[subfigure]{tocloft}

% بسته هایی برای وارد کردن کدهای برنامه نویسی
\usepackage{listings,verbatim}
% بسته‌ای برای ظاهر شدن «مراجع» و «نمایه» در فهرست مطالب و حذف لیست اشکال و جداول از فهرست
\usepackage[nottoc,notlof,notlot]{tocbibind}
% دستورات مربوط به ایجاد نمایه
\usepackage{makeidx}
\makeindex
\usepackage[xindy,acronym,toc,nopostdot]{glossaries}%,style=altlist
%%%%%%%%%%%%%%%%%%%%%%%%%%
% فراخوانی بسته زی‌پرشین و تعریف قلم فارسی و انگلیسی
\usepackage{xepersian}
\settextfont[Scale=1]{XB Zar}
\setlatintextfont[Scale=1]{Times New Roman}

%%%%%%%%%%%%%%%%%%%%%%%%%%
% چنانچه می‌خواهید اعداد در فرمول‌ها، انگلیسی باشد، خط زیر را غیرفعال کنید
\setdigitfont[Scale=1]{XB Zar}%{Persian Modern}
%%%%%%%%%%%%%%%%%%%%%%%%%%
% تعریف قلم‌های فارسی و انگلیسی اضافی برای استفاده در بعضی از قسمت‌های متن
\defpersianfont\titlefont[Scale=1]{XB Titre}
%\defpersianfont\BZar[Scale=1.1]{B Zar}

%%%%%%%%%%%%%%%%%%%%%%%%%%
% دستوری برای حذف کلمه «چکیده»
\renewcommand{\abstractname}{}
% دستوری برای حذف کلمه «abstract»
%\renewcommand{\latinabstract}{}
% دستوری برای تغییر نام کلمه «اثبات» به «برهان»
\renewcommand\proofname{\textbf{برهان}}
% دستوری برای تغییر نام کلمه «کتاب‌نامه» به «مراجع»
\renewcommand{\bibname}{فهرست منابع}
% دستوری برای تعریف واژه‌نامه انگلیسی به فارسی
\newcommand\persiangloss[2]{#1\dotfill\lr{#2}\\}
% دستوری برای تعریف واژه‌نامه فارسی به انگلیسی 
\newcommand\englishgloss[2]{#2\dotfill\lr{#1}\\}
% تعریف دستور جدید «\پ» برای خلاصه‌نویسی جهت نوشتن عبارت «پروژه/پایان‌نامه/رساله»
\newcommand{\پ}{پروژه/پایان‌نامه/رساله }

%\newcommand\BackSlash{\char`\\}

%%%%%%%%%%%%%%%%%%%%%%%%%%
% دستوری برای مشخص کردن کاراکتر بین شماره فصل و بخش
\SepMark{-}

% تعریف و نحوه ظاهر شدن عنوان قضیه‌ها، تعریف‌ها، مثال‌ها و ...
\theoremstyle{definition}
\newtheorem{definition}{تعریف}[section]
\theoremstyle{theorem}
\newtheorem{theorem}[definition]{قضیه}
\newtheorem{lemma}[definition]{لم}
\newtheorem{proposition}[definition]{گزاره}
\newtheorem{corollary}[definition]{نتیجه}
\newtheorem{remark}[definition]{ملاحظه}
\theoremstyle{definition}
\newtheorem{example}[definition]{مثال}

%\renewcommand{\theequation}{\thechapter-\arabic{equation}}
%\def\bibname{فهرست منابع}
\numberwithin{algorithm}{chapter}
\def\listalgorithmname{فهرست الگوریتم‌ها}
\def\listfigurename{فهرست شکل‌ها}
\def\listtablename{فهرست جدول‌ها}

%%%%%%%%%%%%%%%%%%%%%%%%%%%%
% دستورهایی برای سفارشی کردن سربرگ صفحات
\csname@twosidetrue\endcsname
\pagestyle{fancy}
\fancyhf{} 
\fancyhead[RE,LO]{\thepage}
\fancyhead[LE]{\small\leftmark}
% اگر میخواهید در صفحه‌های زوج به جای عنوان فصل، عنوان بخش را داشته باشید به جای leftmark از rightmark استفاده کنید.
\fancyhead[RO]{\small\leftmark}
\renewcommand{\chaptermark}[1]{%
\markboth{
فصل
\tartibi{chapter}~:~#1}{}
} % \thechapter

\doublespacing
%%%%%%%%%%%%%%%%%%%%%%%%%%%%%
% دستوراتی برای اضافه کردن کلمه «فصل» در فهرست مطالب

\newlength\mylenprt
\newlength\mylenchp
\newlength\mylenapp

\renewcommand\cftpartpresnum{\partname~}
\renewcommand\cftchappresnum{\chaptername~}
\renewcommand\cftchapaftersnum{:}

\settowidth\mylenprt{\cftpartfont\cftpartpresnum\cftpartaftersnum}
\settowidth\mylenchp{\cftchapfont\cftchappresnum\cftchapaftersnum}
\settowidth\mylenapp{\cftchapfont\appendixname~\cftchapaftersnum}
\addtolength\mylenprt{\cftpartnumwidth}
\addtolength\mylenchp{\cftchapnumwidth}
\addtolength\mylenapp{\cftchapnumwidth}

\setlength\cftpartnumwidth{\mylenprt}
\setlength\cftchapnumwidth{\mylenchp}	

\makeatletter
{\def\thebibliography#1{\chapter*{\refname\@mkboth
   {\uppercase{\refname}}{\uppercase{\refname}}}\list
   {[\arabic{enumi}]}{\settowidth\labelwidth{[#1]}
   \rightmargin\labelwidth
   \advance\rightmargin\labelsep
   \advance\rightmargin\bibindent
   \itemindent -\bibindent
   \listparindent \itemindent
   \parsep \z@
   \usecounter{enumi}}
   \def\newblock{}
   \sloppy
   \sfcode`\.=1000\relax}}
\makeatother


% دستوری برای تعیین عمق شماره بخش ها. 4 یعنی 1.1.1.1
\setcounter{secnumdepth}{4}
% دستوری برای تعیین عمق شماره بخش ها در فهرست. 2 یعنی 1.1.1
\setcounter{tocdepth}{2}

% دستوری برای افزودن صفحه خالی قبل از شروع فصل ها (تا شماره صفحه شروع هر فصل فرد باشد)
\makeatletter
 \def\cleardoublepage{\clearpage\if@twoside% 
 	\ifodd\c@page\else
 	 \vspace*{\fill} 
 	 \hfill 
 	 \begin{center}
 	 	\begin{Large}
 	 			 	
 	 	\end{Large}
  	\end{center} 
  \vspace{\fill} 
  \thispagestyle{empty} 
  \newpage \if@twocolumn\hbox{}\newpage\fi\fi\fi 
}
 \makeatother
 
% دستوری برای حذف شماره صفحه لیست اشکال، جداول و ...
\makeatletter
\newcommand{\emptypage}[1]{%
%	\cleardoublepage
	\begingroup
	\let\ps@plain\ps@empty
	\pagestyle{empty}
	#1
	\cleardoublepage}
\makeatletter 
% تبدیل <آ> به <الف> در شماره گذاری حرفی (شماره ضمیمه و ...)
\makeatletter
\bidi@patchcmd{\@harfi}{آ}{الف}
{\typeout{Succeeded in changing `آ` into `الف`}}
{\typeout{Failed in changing `آ` into `الف`}}
\makeatother

% فارسی کردن شماره پاورقی های انگلیسی
\makeatletter
\def\LTRfootnote{\@ifnextchar[\@xLTRfootnote{\stepcounter\@mpfn
		\protected@xdef\@thefnmark{\persianfont\thempfn}%
		\@footnotemark\@LTRfootnotetext}}
\makeatother
% دستوراتی برای خارج کردن از بالانویس و اضافه کردن خط تیره به شماره پاورقی
%\makeatletter
%\long\def\@makefntext#1{\parindent 1em
%	\noindent\hbox to 2em{}%
%	\llap{\@thefnmark }-\,\,#1}
%\makeatother

%%%%%%%%%%%%%%%%%%%%%%%%
%%%%%%%%%%%%%%%%%%%%%%%%%%%%
%برای پاورقی شدن کلمات
\defglsentryfmt[english]{\glsgenentryfmt\ifglsused{\glslabel}{}{\LTRfootnote{\glsentryname{\glslabel}}}}
\defglsentryfmt[acronym]{\glsentryname{\glslabel}\ifglsused{\glslabel}{}{\LTRfootnote{\glsentrydesc{\glslabel}}}}
%%%%%%%%%%%%%%%%%%%%%%%%%%%%%%%%%
%تعریف استایل دلخواه    
%%%%%%%%%%%%%%%%%%%%%%%%%%%%%%%%%

\newglossarystyle{mylistFa}{
	\glossarystyle{list}
	\renewenvironment{theglossary}{}{}
	\renewcommand*{\glossaryheader}{}
	\renewcommand*{\glsgroupheading}[1]{\section*{\glsgetgrouptitle{##1}}}
	\renewcommand*{\glsgroupskip}{}
	\renewcommand*{\glossaryentryfield}[5]     {\noindent\glstarget{##1}{##2}\dotfill \space ##3 \\}
	\renewcommand*{\glossarysubentryfield}[6]{\glossaryentryfield{##2}{##3}{##4}{##5}{##6}}
}

\newglossarystyle{mylistEn}{
	\glossarystyle{list}
	\renewenvironment{theglossary}{}{}
	\renewcommand*{\glossaryheader}{}
	\renewcommand*{\glsgroupheading}[1]{\begin{LTR} \section*{\lr{\glsgetgrouptitle{##1}}} \end{LTR}}
	\renewcommand*{\glsgroupskip}{}
	\renewcommand*{\glossaryentryfield}[5]     {\noindent\glstarget{##1}{##3}\dotfill \space ##2 \\}
	\renewcommand*{\glossarysubentryfield}[6]{\glossaryentryfield{##2}{##3}{##4}{##5}{##6}}
}

\newglossary[glg]{english}{gls}{glo}{واژه‌نامه انگلیسی به فارسی}
\newglossary[blg]{persian}{bls}{blo}{واژه‌نامه فارسی به انگلیسی}
\makeglossaries

%%%%%%%%%%%%%%%%%%%%%%%%%%%%%%%%%
%تعریف فرمت واژگان    
%%%%%%%%%%%%%%%%%%%%%%%%%%%%%%%%%
%با این دستور واژه مورد نظر، در متن، در پاورقی و هر دو واژه‌نامه  می‌آید.
\newcommand{\inpdic}[2]{
	\newglossaryentry{fa #1}{type=persian,name={#1}, sort={#1},description={\lr{#2}}}\gls{fa #1}\LTRfootnote{#2}\,\,
	\newglossaryentry{en#1}{type=english,name={\lr{#2}}, sort={#2},description={#1}}\glsuseri{en#1}
	\!\!\!\!}

%%%%%%%%%%%%%%%%%%%%%%%%
%با این دستور واژه مورد نظر، در پاورقی و هر دو واژه‌نامه  می‌آید.
\newcommand{\infdic}[2]{
	\newglossaryentry{fa-#1}{type=persian,name={#1},sort={#1},description={\lr{#2}}}\!\!\glsuseri{fa-#1}\LTRfootnote{#2}
	\newglossaryentry{en-#1}{type=english,name={\lr{#2}}, sort={#2},description={#1}}\glsuseri{en-#1}
	\!\!\!\!}

%%%%%%%%%%%%%%%%%%%%%%%%
% با این دستور واژه مورد نظر در متن و هر دو واژه‌نامه  می‌آید.
\newcommand{\indic}[2]{
	\newglossaryentry{fa-#1}{type=persian,name={#1}, sort={#1},description={\lr{#2}}}\gls{fa-#1}
	\newglossaryentry{en-#1}{type=english,name={\lr{#2}}, sort={#2},description={#1}}\glsuseri{en-#1}
}
%%%%%%%%%%%%%%%%%%%%%%%%
% با این دستور واژه مورد نظر فقط در هر دو واژه‌نامه  می‌آید.
\newcommand{\ingls}[2]{
	\newglossaryentry{fa-#1}{type=persian,name={#1},sort={#1},description={\lr{#2}}}\glsuseri{fa-#1}\!\!\!\!\!
	\newglossaryentry{en-#1}{type=english,name={\lr{#2}},sort={#2},description={#1}}\glsuseri{en-#1}
}
%%%%%%%%%%%%%%%%%%%%%%%%
\allowdisplaybreaks
\setlength{\headheight}{15pt}

%=======================

\lstset{  
	basicstyle=\ttfamily, 
	numbers=left,                   
	numberstyle=\tiny\color{blue},  
	stepnumber=1,                
	numbersep=5pt,                
	backgroundcolor=\color{gray!10},  
	showspaces=false,              
	showstringspaces=false,        
	showtabs=false,                
	rulecolor=\color{black},         
	tabsize=2,                     
	captionpos=b,                   
	breaklines=true,                
	breakatwhitespace=false,        
	keywordstyle=\color{RoyalBlue},      
	commentstyle=\color{gray},  
	stringstyle=\color{red}%{ForestGreen}      
} 

% دستوراتی برای ترتیبی کردن شماره فصل ها در فهرست 

\makeatletter
\newcommand*{\@thechapapp}{\@tartibi\c@chapter}
\bidi@appto\appendix{\gdef\@thechapapp{\@harfi\c@chapter}}

\bidi@patchcmd{\Hy@org@chapter}{%
	\addcontentsline{toc}{chapter}%
	{\protect\numberline{\thechapter}#1}%
}{%
	\addcontentsline{toc}{chapter}%
	{\protect\numberline{\@thechapapp}~#1}%
}{\typeout{We succeded in redefining \string\@chapter}}
{\typeout{We failed in redefining \string\@chapter}}
