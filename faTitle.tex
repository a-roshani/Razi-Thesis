% !TeX root=main.tex
% در این فایل، عنوان پایان‌نامه، مشخصات خود، متن تقدیمی‌، ستایش، سپاس‌گزاری و چکیده پایان‌نامه را به فارسی، وارد کنید.
% توجه داشته باشید که جدول حاوی مشخصات پروژه/پایان‌نامه/رساله و همچنین، مشخصات داخل آن، به طور خودکار، درج می‌شود.
%%%%%%%%%%%%%%%%%%%%%%%%%%%%%%%%%%%%
% دانشگاه خود را وارد کنید
\university{دانشگاه رازی}
% دانشکده، آموزشکده و یا پژوهشکده  خود را وارد کنید
\faculty{دانشکده علوم}
% گروه آموزشی خود را وارد کنید
\department{گروه آمار}
% گروه آموزشی خود را وارد کنید
\subject{آمار}
% گرایش خود را وارد کنید
\field{ریاضی}
% اگر دانشجوی کارشناسی هستید و گرایش ندارید، field را در قسمت قبل خالی بگذارید و عبارت گرایش را در دستور زیر پاک کنید یعنی به صورت زیر
%\field{}
%\def\gerayesh{}

\def\gerayesh{گرایش}
\def\reshte{رشته}
% عنوان پایان‌نامه را وارد کنید
\title{
	شیوه‌نامه نوشتن پروژه، پایان‌نامه و رساله با استفاده از کلاس
\lr{Razi-Thesis}
}
% نام استاد(ان) راهنما را وارد کنید. در صورتی که یک استاد راهنما دارید، استاد راهنمای  دوم را با قرار دادن %  قبل از آن غیر فعال کنید
\firstsupervisor{استاد راهنمای اول}
\secondsupervisor{استاد راهنمای دوم}
% نام استاد(دان) مشاور را وارد کنید. چنانچه استاد مشاور ندارید، دستور پایین را غیرفعال کنید.
\firstadvisor{استاد مشاور اول}
\secondadvisor{استاد مشاور دوم}
% نام دانشجو را وارد کنید
\name{امین}
% نام خانوادگی دانشجو را وارد کنید
\surname{روشنی}
% شماره دانشجویی دانشجو را وارد کنید
\studentID{955125002}
% تاریخ پایان‌نامه را وارد کنید
\thesisdate{شهریور 1401}
% به صورت پیش‌فرض برای پایان‌نامه‌های کارشناسی تا دکترا به ترتیب از عبارات «پروژه»، «پایان‌نامه» و »رساله» استفاده می‌شود؛ اگر  نمی‌پسندید هر عنوانی را که مایلید در دستور زیر قرار داده و آنرا از حالت توضیح خارج کنید.
%\projectLabel{پایان‌نامه}

% به صورت پیش‌فرض برای عناوین مقاطع تحصیلی کارشناسی تا دکترا به ترتیب از عبارات «کارشناسی»، «کارشناسی ارشد» و »دکترا» استفاده می‌شود؛ اگر  نمی‌پسندید هر عنوانی را که مایلید در دستور زیر قرار داده و آنرا از حالت توضیح خارج کنید.
%\degree{کارشناسی ارشد}
\besmPage
{\null\newpage}
\newpage
\firstPage
\newpage
\Esalat
\newpage
\thispagestyle{empty}
\Rights
\davaranPage
%\newpage
%\thispagestyle{empty}
%{\null\newpage}
\SogandName
%\newpage
%\thispagestyle{empty}
%{\null\newpage}


% چنانچه مایل به چاپ صفحات «تقدیم»، «نیایش» و «سپاس‌گزاری» در خروجی نیستید، خط‌های زیر را با گذاشتن ٪  در ابتدای آنها غیرفعال کنید.
 % پایان‌نامه خود را تقدیم کنید!

\newpage
\begin{taghdim}
\vspace{3cm}
\begin{center}
{\huge\titlefont
		پدر و مادرم
}
\end{center}
\end{taghdim}	

% سپاس‌گزاری
\begin{acknowledgementpage}
سپاس خداوندگار حکیم را که با لطف بی‌کران خود، آدمی را زیور عقل آراست.


در آغاز وظیفه‌  خود  می‌دانم از زحمات بی‌دریغ استاد  راهنمای خود،  جناب آقای دکتر ...، صمیمانه تشکر و  قدردانی کنم  که قطعاً بدون راهنمایی‌های ارزنده‌  ایشان، این مجموعه  به انجام  نمی‌رسید.

از جناب  آقای  دکتر ...   که زحمت  مطالعه و مشاوره‌  این رساله را تقبل  فرمودند و در آماده سازی  این رساله، به نحو احسن اینجانب را مورد راهنمایی قرار دادند، کمال امتنان را دارم.

 در پایان، بوسه می‌زنم بر دستان خداوندگاران مهر و مهربانی، پدر و مادر عزیزم و بعد از خدا، ستایش می‌کنم وجود مقدس‌شان را و تشکر می‌کنم از خانواده عزیزم به پاس عاطفه سرشار و گرمای امیدبخش وجودشان، که بهترین پشتیبان من بودند.
% با استفاده از دستور زیر، امضای شما، به طور خودکار، درج می‌شود.
\signature 
\end{acknowledgementpage}
%%%%%%%%%%%%%%%%%%%%%%%%%%%%%%%%%%%%
% کلمات کلیدی پایان‌نامه را وارد کنید
\keywords{زی‌پرشین، لاتک، قالب پایان‌نامه، الگو}
%چکیده پایان‌نامه را وارد کنید، برای ایجاد پاراگراف جدید از \\ استفاده کنید. اگر خط خالی دشته باشید، خطا خواهید گرفت.
\fa-abstract{
این پایان‌نامه، به بحث در مورد نوشتن پروژه، پایان‌نامه و رساله با استفاده از کلاس 
\lr{Razi-Thesis}
می‌پردازد.
حروف‌چینی پروژه کارشناسی، پایان‌نامه یا رساله یکی از موارد پرکاربرد استفاده از زی‌پرشین است. 
زی‌پرشین بسته‌ای است که به همت آقای وفا خلیقی آماده شده است و امکان حروف‌چینی فارسی در \lr{\LaTeXe}{} را  برای فارسی‌زبانان فراهم کرده است.
از جمله مزایای لاتک آن است که در صورت وجود یک کلاس آماده برای حروف‌چینی یک سند خاص مانند یک پایان‌نامه، کاربر بدون درگیری با جزییات حروف‌چینی و صفحه‌آرایی می‌تواند سند خود را آماده نماید.
\\
\indent
شاید با قالب‌های لاتکی که برخی از مجلات برای مقالات خود عرضه می‌کنند مواجه شده باشید. اگر نظیر این کار در دانشگاههای مختلف برای اسناد متنوع آنها مانند پایا‌ن‌نامه‌ها آماده شود، دانشجویان به جای وقت گذاشتن روی صفحه‌آرایی مطالب خود، روی محتوای متن خود تمرکز خواهند نمود. به علاوه با آشنایی با لاتک خواهند توانست از امکانات بسیار این نرم‌افزار جهت نمایش بهتر دست‌آوردهای خود استفاده کنند.
به همین خاطر، یک کلاس با نام 
\lr{Razi-Thesis}
 برای حروف‌چینی پروژه‌ها، پایان‌نامه‌ها و رساله‌های دانشگاه لرستان با استفاده از نرم‌افزار لاتک و بسته زی‌پرشین،  آماده شده است. این فایل به 
گونه‌ای طراحی شده است که کلیات خواسته‌های مورد نیاز  مدیریت تحصیلات تکمیلی دانشگاه لرستان را برآورده می‌کند و نیز، حروف‌چینی بسیاری از قسمت‌های آن، به طور خودکار انجام می‌شود.
}

\abstractPage
% اگر می‌خواهید پیشگفتار داشته باشید چند خط زیر را فعال کنید.
%\preface{
%	در این قسمت متن پیش‌گفتار را وارد کنید...
%}
%\prefacePage

\newpage\clearpage