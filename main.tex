
%        نمونه پایان‌نامه آماده شده با استفاده از کلاس Razi-Thesis، نگارش 0.1
% 	     امین روشنی، دانشگاه رازی، دانشکده علوم، گروه آمار
%        این نسخه، بر اساس نسخه‌ 0.4 از کلاس Tabriz_Thesis آقای وحید دامن‌افشان آماده شده است. http://damanafshan.tk        
%---------------------------------------------------------------------------------------------
%        اگر قصد نوشتن پروژه کارشناسی را دارید، در خط زیر به جای msc، کلمه bsc و اگر قصد نوشتن پروژه دکترا
%        را دارید، کلمه phd را قرار دهید. کلیه تنظیمات لازم، به طور خودکار، اعمال می‌شود.
\documentclass[phd]{Razi-Thesis}

% مشخصات دانشجو و پایان‌نامه، سوگندنامه، برگ اصالت و مالکیت اثر، تشکر و قدردانی را در فایلهای faTitle و enTitle وارد نمایید.

%       فایل commands.tex را مطالعه کنید؛ چون دستورات مربوط به فراخوانی بسته زی‌پرشین 
%       و دیگر بسته‌ها و ... در این فایل قرار دارد و بهتر است که با نحوه استفاده از آنها آشنا شوید.
% در این فایل، دستورها و تنظیمات مورد نیاز، آورده شده است.
%-------------------------------------------------------------------------------------------------------------------

% در ورژن جدید زی‌پرشین برای تایپ متن‌های ریاضی، این سه بسته، حتماً باید فراخوانی شود
\usepackage{amsthm,amssymb,amsmath}
% بسته‌ای برای تنطیم حاشیه‌های بالا، پایین، چپ و راست صفحه
\usepackage[top=25mm, bottom=25mm, left=22mm, right=22mm]{geometry}
% بسته‌‌ای برای ظاهر شدن شکل‌ها و تصاویر متن
\usepackage{graphicx,xcolor}
% بسته‌ای برای رسم کادر
\usepackage{framed} 
% بسته‌‌ای برای چاپ شدن خودکار تعداد صفحات در صفحه «معرفی پایان‌نامه»
\usepackage{lastpage}
% بسته‌ و دستوراتی برای ایجاد لینک‌های رنگی با امکان جهش
\usepackage[linktocpage=true,colorlinks,citecolor=blue,linkcolor=blue]{hyperref}
% چنانچه قصد پرینت گرفتن نوشته خود را دارید، خط بالا را غیرفعال و  از دستور زیر استفاده کنید چون در صورت استفاده از دستور زیر‌‌، 
% لینک‌ها به رنگ سیاه ظاهر خواهند شد که برای پرینت گرفتن، مناسب‌تر است
%\usepackage[pagebackref=false]{hyperref}
% بسته‌ لازم برای تنظیم سربرگ‌ها
\usepackage{fancyhdr}
\usepackage{setspace}
\usepackage{alltt}
\usepackage{zref-perpage}
\usepackage{algorithm}
\usepackage{algorithmic}
\usepackage{subfigure}
\usepackage{natbib}
%\usepackage{bibunits}
\usepackage{bibentry}
\usepackage{titlesec}
\usepackage[subfigure]{tocloft}

% بسته هایی برای وارد کردن کدهای برنامه نویسی
\usepackage{listings,verbatim}
% بسته‌ای برای ظاهر شدن «مراجع» و «نمایه» در فهرست مطالب و حذف لیست اشکال و جداول از فهرست
\usepackage[nottoc,notlof,notlot]{tocbibind}
% دستورات مربوط به ایجاد نمایه
\usepackage{makeidx}
\makeindex
\usepackage[xindy,acronym,toc,nopostdot]{glossaries}%,style=altlist
%%%%%%%%%%%%%%%%%%%%%%%%%%
% فراخوانی بسته زی‌پرشین و تعریف قلم فارسی و انگلیسی
\usepackage{xepersian}
\settextfont[Scale=1]{XB Zar}
\setlatintextfont[Scale=1]{Times New Roman}

%%%%%%%%%%%%%%%%%%%%%%%%%%
% چنانچه می‌خواهید اعداد در فرمول‌ها، انگلیسی باشد، خط زیر را غیرفعال کنید
\setdigitfont[Scale=1]{XB Zar}%{Persian Modern}
%%%%%%%%%%%%%%%%%%%%%%%%%%
% تعریف قلم‌های فارسی و انگلیسی اضافی برای استفاده در بعضی از قسمت‌های متن
\defpersianfont\titlefont[Scale=1]{XB Titre}
%\defpersianfont\BZar[Scale=1.1]{B Zar}

%%%%%%%%%%%%%%%%%%%%%%%%%%
% دستوری برای حذف کلمه «چکیده»
\renewcommand{\abstractname}{}
% دستوری برای حذف کلمه «abstract»
%\renewcommand{\latinabstract}{}
% دستوری برای تغییر نام کلمه «اثبات» به «برهان»
\renewcommand\proofname{\textbf{برهان}}
% دستوری برای تغییر نام کلمه «کتاب‌نامه» به «مراجع»
\renewcommand{\bibname}{فهرست منابع}
% دستوری برای تعریف واژه‌نامه انگلیسی به فارسی
\newcommand\persiangloss[2]{#1\dotfill\lr{#2}\\}
% دستوری برای تعریف واژه‌نامه فارسی به انگلیسی 
\newcommand\englishgloss[2]{#2\dotfill\lr{#1}\\}
% تعریف دستور جدید «\پ» برای خلاصه‌نویسی جهت نوشتن عبارت «پروژه/پایان‌نامه/رساله»
\newcommand{\پ}{پروژه/پایان‌نامه/رساله }

%\newcommand\BackSlash{\char`\\}

%%%%%%%%%%%%%%%%%%%%%%%%%%
% دستوری برای مشخص کردن کاراکتر بین شماره فصل و بخش
\SepMark{-}
% دستوراتی برای تنظیم فاصله بین شماره بخش و زیربخش با عنوان آن در فهرست مطالب. این بخش برای بخض ضمیمه ها که عنوان بخش ها به صورت الف-1 است بیشتر کاربرد دارد
\setlength{\cftsecnumwidth}{2.5em}
\setlength{\cftsubsecnumwidth}{4em}

% تعریف و نحوه ظاهر شدن عنوان قضیه‌ها، تعریف‌ها، مثال‌ها و ...
\theoremstyle{definition}
\newtheorem{definition}{تعریف}[section]
\theoremstyle{theorem}
\newtheorem{theorem}[definition]{قضیه}
\newtheorem{lemma}[definition]{لم}
\newtheorem{proposition}[definition]{گزاره}
\newtheorem{corollary}[definition]{نتیجه}
\newtheorem{remark}[definition]{ملاحظه}
\theoremstyle{definition}
\newtheorem{example}[definition]{مثال}

%\renewcommand{\theequation}{\thechapter-\arabic{equation}}
%\def\bibname{فهرست منابع}
\numberwithin{algorithm}{chapter}
\def\listalgorithmname{فهرست الگوریتم‌ها}
\def\listfigurename{فهرست شکل‌ها}
\def\listtablename{فهرست جدول‌ها}

%%%%%%%%%%%%%%%%%%%%%%%%%%%%
% دستورهایی برای سفارشی کردن سربرگ صفحات
\csname@twosidetrue\endcsname
\pagestyle{fancy}
\fancyhf{} 
\fancyhead[RE,LO]{\thepage}
\fancyhead[LE]{\small\leftmark}
% اگر میخواهید در صفحه‌های زوج به جای عنوان فصل، عنوان بخش را داشته باشید به جای leftmark از rightmark استفاده کنید.
\fancyhead[RO]{\small\leftmark}
\renewcommand{\chaptermark}[1]{%
\markboth{
فصل
\tartibi{chapter}~:~#1}{}
} % \thechapter

\doublespacing
%%%%%%%%%%%%%%%%%%%%%%%%%%%%%
% دستوراتی برای اضافه کردن کلمه «فصل» در فهرست مطالب

\newlength\mylenprt
\newlength\mylenchp
\newlength\mylenapp

\renewcommand\cftpartpresnum{\partname~}
\renewcommand\cftchappresnum{\chaptername~}
\renewcommand\cftchapaftersnum{:}

\settowidth\mylenprt{\cftpartfont\cftpartpresnum\cftpartaftersnum}
\settowidth\mylenchp{\cftchapfont\cftchappresnum\cftchapaftersnum}
\settowidth\mylenapp{\cftchapfont\appendixname~\cftchapaftersnum}
\addtolength\mylenprt{\cftpartnumwidth}
\addtolength\mylenchp{\cftchapnumwidth}
\addtolength\mylenapp{\cftchapnumwidth}

\setlength\cftpartnumwidth{\mylenprt}
\setlength\cftchapnumwidth{\mylenchp}	

\makeatletter
{\def\thebibliography#1{\chapter*{\refname\@mkboth
   {\uppercase{\refname}}{\uppercase{\refname}}}\list
   {[\arabic{enumi}]}{\settowidth\labelwidth{[#1]}
   \rightmargin\labelwidth
   \advance\rightmargin\labelsep
   \advance\rightmargin\bibindent
   \itemindent -\bibindent
   \listparindent \itemindent
   \parsep \z@
   \usecounter{enumi}}
   \def\newblock{}
   \sloppy
   \sfcode`\.=1000\relax}}
\makeatother


% دستوری برای تعیین عمق شماره بخش ها. 4 یعنی 1.1.1.1
\setcounter{secnumdepth}{4}
% دستوری برای تعیین عمق شماره بخش ها در فهرست. 2 یعنی 1.1.1
\setcounter{tocdepth}{2}

% دستوری برای افزودن صفحه خالی قبل از شروع فصل ها (تا شماره صفحه شروع هر فصل فرد باشد)
\makeatletter
 \def\cleardoublepage{\clearpage\if@twoside% 
 	\ifodd\c@page\else
 	 \vspace*{\fill} 
 	 \hfill 
 	 \begin{center}
 	 	\begin{Large}
 	 			 	
 	 	\end{Large}
  	\end{center} 
  \vspace{\fill} 
  \thispagestyle{empty} 
  \newpage \if@twocolumn\hbox{}\newpage\fi\fi\fi 
}
 \makeatother
 
% دستوری برای حذف شماره صفحه لیست اشکال، جداول و ...
\makeatletter
\newcommand{\emptypage}[1]{%
%	\cleardoublepage
	\begingroup
	\let\ps@plain\ps@empty
	\pagestyle{empty}
	#1
	\cleardoublepage}
\makeatletter 
% تبدیل <آ> به <الف> در شماره گذاری حرفی (شماره ضمیمه و ...)
\makeatletter
\bidi@patchcmd{\@harfi}{آ}{الف}
{\typeout{Succeeded in changing `آ` into `الف`}}
{\typeout{Failed in changing `آ` into `الف`}}
\makeatother

% فارسی کردن شماره پاورقی های انگلیسی
\makeatletter
\def\LTRfootnote{\@ifnextchar[\@xLTRfootnote{\stepcounter\@mpfn
		\protected@xdef\@thefnmark{\persianfont\thempfn}%
		\@footnotemark\@LTRfootnotetext}}
\makeatother
% دستوراتی برای خارج کردن از بالانویس و اضافه کردن خط تیره به شماره پاورقی
%\makeatletter
%\long\def\@makefntext#1{\parindent 1em
%	\noindent\hbox to 2em{}%
%	\llap{\@thefnmark }-\,\,#1}
%\makeatother

%%%%%%%%%%%%%%%%%%%%%%%%
%%%%%%%%%%%%%%%%%%%%%%%%%%%%
%برای پاورقی شدن کلمات
\defglsentryfmt[english]{\glsgenentryfmt\ifglsused{\glslabel}{}{\LTRfootnote{\glsentryname{\glslabel}}}}
\defglsentryfmt[acronym]{\glsentryname{\glslabel}\ifglsused{\glslabel}{}{\LTRfootnote{\glsentrydesc{\glslabel}}}}
%%%%%%%%%%%%%%%%%%%%%%%%%%%%%%%%%
%تعریف استایل دلخواه    
%%%%%%%%%%%%%%%%%%%%%%%%%%%%%%%%%

\newglossarystyle{mylistFa}{
	\glossarystyle{list}
	\renewenvironment{theglossary}{}{}
	\renewcommand*{\glossaryheader}{}
	\renewcommand*{\glsgroupheading}[1]{\section*{\glsgetgrouptitle{##1}}}
	\renewcommand*{\glsgroupskip}{}
	\renewcommand*{\glossaryentryfield}[5]     {\noindent\glstarget{##1}{##2}\dotfill \space ##3 \\}
	\renewcommand*{\glossarysubentryfield}[6]{\glossaryentryfield{##2}{##3}{##4}{##5}{##6}}
}

\newglossarystyle{mylistEn}{
	\glossarystyle{list}
	\renewenvironment{theglossary}{}{}
	\renewcommand*{\glossaryheader}{}
	\renewcommand*{\glsgroupheading}[1]{\begin{LTR} \section*{\lr{\glsgetgrouptitle{##1}}} \end{LTR}}
	\renewcommand*{\glsgroupskip}{}
	\renewcommand*{\glossaryentryfield}[5]     {\noindent\glstarget{##1}{##3}\dotfill \space ##2 \\}
	\renewcommand*{\glossarysubentryfield}[6]{\glossaryentryfield{##2}{##3}{##4}{##5}{##6}}
}

\newglossary[glg]{english}{gls}{glo}{واژه‌نامه انگلیسی به فارسی}
\newglossary[blg]{persian}{bls}{blo}{واژه‌نامه فارسی به انگلیسی}
\makeglossaries

%%%%%%%%%%%%%%%%%%%%%%%%%%%%%%%%%
%تعریف فرمت واژگان    
%%%%%%%%%%%%%%%%%%%%%%%%%%%%%%%%%
%با این دستور واژه مورد نظر، در متن، در پاورقی و هر دو واژه‌نامه  می‌آید.
\newcommand{\inpdic}[2]{
	\newglossaryentry{fa #1}{type=persian,name={#1}, sort={#1},description={\lr{#2}}}\gls{fa #1}\LTRfootnote{#2}\,\,
	\newglossaryentry{en#1}{type=english,name={\lr{#2}}, sort={#2},description={#1}}\glsuseri{en#1}
	\!\!\!\!}

%%%%%%%%%%%%%%%%%%%%%%%%
%با این دستور واژه مورد نظر، در پاورقی و هر دو واژه‌نامه  می‌آید.
\newcommand{\infdic}[2]{
	\newglossaryentry{fa-#1}{type=persian,name={#1},sort={#1},description={\lr{#2}}}\!\!\glsuseri{fa-#1}\LTRfootnote{#2}
	\newglossaryentry{en-#1}{type=english,name={\lr{#2}}, sort={#2},description={#1}}\glsuseri{en-#1}
	\!\!\!\!}

%%%%%%%%%%%%%%%%%%%%%%%%
% با این دستور واژه مورد نظر در متن و هر دو واژه‌نامه  می‌آید.
\newcommand{\indic}[2]{
	\newglossaryentry{fa-#1}{type=persian,name={#1}, sort={#1},description={\lr{#2}}}\gls{fa-#1}
	\newglossaryentry{en-#1}{type=english,name={\lr{#2}}, sort={#2},description={#1}}\glsuseri{en-#1}
}
%%%%%%%%%%%%%%%%%%%%%%%%
% با این دستور واژه مورد نظر فقط در هر دو واژه‌نامه  می‌آید.
\newcommand{\ingls}[2]{
	\newglossaryentry{fa-#1}{type=persian,name={#1},sort={#1},description={\lr{#2}}}\glsuseri{fa-#1}\!\!\!\!\!
	\newglossaryentry{en-#1}{type=english,name={\lr{#2}},sort={#2},description={#1}}\glsuseri{en-#1}
}
%%%%%%%%%%%%%%%%%%%%%%%%
\allowdisplaybreaks
\setlength{\headheight}{15pt}

%=======================

\lstset{  
	basicstyle=\ttfamily, 
	numbers=left,                   
	numberstyle=\tiny\color{blue},  
	stepnumber=1,                
	numbersep=5pt,                
	backgroundcolor=\color{gray!10},  
	showspaces=false,              
	showstringspaces=false,        
	showtabs=false,                
	rulecolor=\color{black},         
	tabsize=2,                     
	captionpos=b,                   
	breaklines=true,                
	breakatwhitespace=false,        
	keywordstyle=\color{RoyalBlue},      
	commentstyle=\color{gray},  
	stringstyle=\color{red}%{ForestGreen}      
} 

% دستوراتی برای ترتیبی کردن شماره فصل ها در فهرست 

\makeatletter
\newcommand*{\@thechapapp}{\@tartibi\c@chapter}
\bidi@appto\appendix{\gdef\@thechapapp{\@harfi\c@chapter}}

\bidi@patchcmd{\Hy@org@chapter}{%
	\addcontentsline{toc}{chapter}%
	{\protect\numberline{\thechapter}#1}%
}{%
	\addcontentsline{toc}{chapter}%
	{\protect\numberline{\@thechapapp}~#1}%
}{\typeout{We succeded in redefining \string\@chapter}}
{\typeout{We failed in redefining \string\@chapter}}


\begin{document}
% اگر میخواهید شماره صفحه‌ها در بخش های آغازین (فهرست و ...) به صورت حروف الفبای فارسی ظاهر شود کد زیر را فعال کنید.
%\pagenumbering{harfi}
% !TeX root=main.tex
% در این فایل، عنوان پایان‌نامه، مشخصات خود، متن تقدیمی‌، ستایش، سپاس‌گزاری و چکیده پایان‌نامه را به فارسی، وارد کنید.
% توجه داشته باشید که جدول حاوی مشخصات پروژه/پایان‌نامه/رساله و همچنین، مشخصات داخل آن، به طور خودکار، درج می‌شود.
%%%%%%%%%%%%%%%%%%%%%%%%%%%%%%%%%%%%
% دانشگاه خود را وارد کنید
\university{دانشگاه رازی}
% دانشکده، آموزشکده و یا پژوهشکده  خود را وارد کنید
\faculty{دانشکده علوم}
% گروه آموزشی خود را وارد کنید
\department{گروه آمار}
% گروه آموزشی خود را وارد کنید
\subject{آمار}
% گرایش خود را وارد کنید
\field{ریاضی}
% اگر دانشجوی کارشناسی هستید و گرایش ندارید، field را در قسمت قبل خالی بگذارید و عبارت گرایش را در دستور زیر پاک کنید یعنی به صورت زیر
%\field{}
%\def\gerayesh{}

\def\gerayesh{گرایش}
\def\reshte{رشته}
% عنوان پایان‌نامه را وارد کنید
\title{
	شیوه‌نامه نوشتن پروژه، پایان‌نامه و رساله با استفاده از کلاس
\lr{Razi-Thesis}
}
% نام استاد(ان) راهنما را وارد کنید. در صورتی که یک استاد راهنما دارید، استاد راهنمای  دوم را با قرار دادن %  قبل از آن غیر فعال کنید
\firstsupervisor{استاد راهنمای اول}
\secondsupervisor{استاد راهنمای دوم}
% نام استاد(دان) مشاور را وارد کنید. چنانچه استاد مشاور ندارید، دستور پایین را غیرفعال کنید.
\firstadvisor{استاد مشاور اول}
\secondadvisor{استاد مشاور دوم}
% نام دانشجو را وارد کنید
\name{امین}
% نام خانوادگی دانشجو را وارد کنید
\surname{روشنی}
% شماره دانشجویی دانشجو را وارد کنید
\studentID{955125002}
% تاریخ پایان‌نامه را وارد کنید
\thesisdate{شهریور 1401}
% به صورت پیش‌فرض برای پایان‌نامه‌های کارشناسی تا دکترا به ترتیب از عبارات «پروژه»، «پایان‌نامه» و »رساله» استفاده می‌شود؛ اگر  نمی‌پسندید هر عنوانی را که مایلید در دستور زیر قرار داده و آنرا از حالت توضیح خارج کنید.
%\projectLabel{پایان‌نامه}

% به صورت پیش‌فرض برای عناوین مقاطع تحصیلی کارشناسی تا دکترا به ترتیب از عبارات «کارشناسی»، «کارشناسی ارشد» و »دکترا» استفاده می‌شود؛ اگر  نمی‌پسندید هر عنوانی را که مایلید در دستور زیر قرار داده و آنرا از حالت توضیح خارج کنید.
%\degree{کارشناسی ارشد}
\besmPage
{\null\newpage}
\newpage
\firstPage
\newpage
\Esalat
\newpage
\thispagestyle{empty}
\Rights
\davaranPage
%\newpage
%\thispagestyle{empty}
%{\null\newpage}
\SogandName
%\newpage
%\thispagestyle{empty}
%{\null\newpage}


% چنانچه مایل به چاپ صفحات «تقدیم»، «نیایش» و «سپاس‌گزاری» در خروجی نیستید، خط‌های زیر را با گذاشتن ٪  در ابتدای آنها غیرفعال کنید.
 % پایان‌نامه خود را تقدیم کنید!

\newpage
\begin{taghdim}
\vspace{3cm}
\begin{center}
{\huge\titlefont
		پدر و مادرم
}
\end{center}
\end{taghdim}	

% سپاس‌گزاری
\begin{acknowledgementpage}
سپاس خداوندگار حکیم را که با لطف بی‌کران خود، آدمی را زیور عقل آراست.


در آغاز وظیفه‌  خود  می‌دانم از زحمات بی‌دریغ استاد  راهنمای خود،  جناب آقای دکتر ...، صمیمانه تشکر و  قدردانی کنم  که قطعاً بدون راهنمایی‌های ارزنده‌  ایشان، این مجموعه  به انجام  نمی‌رسید.

از جناب  آقای  دکتر ...   که زحمت  مطالعه و مشاوره‌  این رساله را تقبل  فرمودند و در آماده سازی  این رساله، به نحو احسن اینجانب را مورد راهنمایی قرار دادند، کمال امتنان را دارم.

 در پایان، بوسه می‌زنم بر دستان خداوندگاران مهر و مهربانی، پدر و مادر عزیزم و بعد از خدا، ستایش می‌کنم وجود مقدس‌شان را و تشکر می‌کنم از خانواده عزیزم به پاس عاطفه سرشار و گرمای امیدبخش وجودشان، که بهترین پشتیبان من بودند.
% با استفاده از دستور زیر، امضای شما، به طور خودکار، درج می‌شود.
\signature 
\end{acknowledgementpage}
%%%%%%%%%%%%%%%%%%%%%%%%%%%%%%%%%%%%
% کلمات کلیدی پایان‌نامه را وارد کنید
\keywords{زی‌پرشین، لاتک، قالب پایان‌نامه، الگو}
%چکیده پایان‌نامه را وارد کنید، برای ایجاد پاراگراف جدید از \\ استفاده کنید. اگر خط خالی دشته باشید، خطا خواهید گرفت.
\fa-abstract{
این پایان‌نامه، به بحث در مورد نوشتن پروژه، پایان‌نامه و رساله با استفاده از کلاس 
\lr{Razi-Thesis}
می‌پردازد.
حروف‌چینی پروژه کارشناسی، پایان‌نامه یا رساله یکی از موارد پرکاربرد استفاده از زی‌پرشین است. 
زی‌پرشین بسته‌ای است که به همت آقای وفا خلیقی آماده شده است و امکان حروف‌چینی فارسی در \lr{\LaTeXe}{} را  برای فارسی‌زبانان فراهم کرده است.
از جمله مزایای لاتک آن است که در صورت وجود یک کلاس آماده برای حروف‌چینی یک سند خاص مانند یک پایان‌نامه، کاربر بدون درگیری با جزییات حروف‌چینی و صفحه‌آرایی می‌تواند سند خود را آماده نماید.
\\
\indent
شاید با قالب‌های لاتکی که برخی از مجلات برای مقالات خود عرضه می‌کنند مواجه شده باشید. اگر نظیر این کار در دانشگاههای مختلف برای اسناد متنوع آنها مانند پایا‌ن‌نامه‌ها آماده شود، دانشجویان به جای وقت گذاشتن روی صفحه‌آرایی مطالب خود، روی محتوای متن خود تمرکز خواهند نمود. به علاوه با آشنایی با لاتک خواهند توانست از امکانات بسیار این نرم‌افزار جهت نمایش بهتر دست‌آوردهای خود استفاده کنند.
به همین خاطر، یک کلاس با نام 
\lr{Razi-Thesis}
 برای حروف‌چینی پروژه‌ها، پایان‌نامه‌ها و رساله‌های دانشگاه لرستان با استفاده از نرم‌افزار لاتک و بسته زی‌پرشین،  آماده شده است. این فایل به 
گونه‌ای طراحی شده است که کلیات خواسته‌های مورد نیاز  مدیریت تحصیلات تکمیلی دانشگاه لرستان را برآورده می‌کند و نیز، حروف‌چینی بسیاری از قسمت‌های آن، به طور خودکار انجام می‌شود.
}

\abstractPage
% اگر می‌خواهید پیشگفتار داشته باشید چند خط زیر را فعال کنید.
%\preface{
%	در این قسمت متن پیش‌گفتار را وارد کنید...
%}
%\prefacePage

\newpage\clearpage
\thispagestyle{empty}
\emptypage\tableofcontents
% فهرست علائم
\thispagestyle{empty}

{\noindent \bfseries \Huge
	فهرست نشانه‌ها و نمادها
}

%\addcontentsline{toc}{chapter}{فهرست علائم}

\section*{نشانه‌ها}
\persiangloss{میلی‌لیتر}{\lr{ml}}
\section*{نمادها}
\persiangloss{انحراف معیار}{$\sigma$}
\section*{کوته‌نوشت‌ها}
\persiangloss{اچ‌تی‌ام‌ال}{\lr{HTML}}




\newpage
% فهرست جدول‌ها
\listoftables  
\newpage
% فهرست شکل‌ها
\listoffigures 
\newpage
% فهرست الگوریتم ها و اضافه کردن آن به فهرست مطالب
%\addcontentsline{toc}{chapter}{\listalgorithmname}
%\listofalgorithms \newpage

%%%%%%%%%%%%%%%%%%%%%%%%
%TODOدستورات مربوط به صفحات اول هر فصل
{
	\titleformat{\chapter}[display]
	{\vspace{-3cm}\vfill\filcenter}
	{{%
			\thispagestyle{empty}
			\vspace{-3cm}
			\filcenter\fontsize{31pt}{31pt}\selectfont
			{	\bfseries{\chaptername}}
			\fontsize{31pt}{31pt}\selectfont\bfseries{\tartibi{chapter}}%
		}%
	}
	{40pt}
	{\fontsize{33pt}{33pt}\selectfont%
	}[\vfill\clearpage]
	{\bfseries\titlespacing*{\chapter}{0pt}{0pt}{0pt}}

\pagestyle{fancy}
% اگر شما فصل اول  خود را در فایلی به جز chapter1 همراه با این کلاس نوشته‌اید باید چندخط اول chapter1 را در فایل خود کپی کنید.
\include{chapters/chapter1}			% فصل اول: مقدمه
\include{chapters/chapter2}		% فصل دوم: آشنایی مقدماتی با لاتک
% فصل سوم
\include{chapters/chapter3}

}
% فهرست منابع
\pagestyle{fancy}
{
\onehalfspacing
\bibliographystyle{chicago-fa}%{acm-fa}{chicago-fa}%{plainnat-fa}%
\bibliography{references/MyReferences}
}

\pagestyle{fancy}

\appendix                           %فصلهای پس از این قسمت به عنوان ضمیمه خواهند آمد.
% اگر شما پیوست اول  خود را در فایلی به جز appendix1 همراه با این کلاس نوشته‌اید باید چندخط اول appendix1 را در فایل خود کپی کنید.
\include{chapters/appendix1}		% پیوست اول: مدیریت مراجع در لاتک
\include{chapters/appendix2}
\include{chapters/appendix3}


%\baselineskip=.75cm
%\onehalfspacing
%\include{dicfa2en}
%\include{dicen2fa}
% برای آموزش نحوه خروجی گرفتن از واژه نامه‌ها به لینک زیر مراجعه کنید
% https://tex.stackexchange.com/questions/448387/glossary-and-abbreviation-showing-problem-in-xepersianshown-in-blank-pages
\newpage
\markboth{واژه‌نامه فارسی به انگلیسی}{واژه‌نامه فارسی به انگلیسی}
\printglossary[type=persian,style=mylistFa]
\newpage
\markboth{واژه‌نامه انگلیسی به فارسی}{واژه‌نامه انگلیسی به فارسی}
\printglossary[type=english,style=mylistEn]

% فهرست مقاله‌های برگرفته از پارسا. اگر مقاله ای ندارید سه خط زیر را غیر فعال کنید.
\newpage
\thispagestyle{empty}
\thispagestyle{empty}
\makeatletter 
\renewcommand\BR@b@bibitem[2][]{\BR@bibitem[#1]{#2}\BR@c@bibitem{#2}}           
\makeatother
\nobibliography*
%  ترتیب مراجع در خروجی بر اساس ترتیب وارد کردن آن‌ها است. 
\begin{publicationspage}
\begin{latin}
\noindent
\bibentry{Baker02limits}\\[0.5cm]
\bibentry{Amintoosi09precise}
\end{latin}
\end{publicationspage}

% بخش کارنامک برای دانش‌آموختگان مقطع دکتری اجباری و برای مقطع کارشناسی ارشد اختیاری است. در صورتی که دانشجوی مقطع دکتری نیستید و نمی‌خواهید کارنامک داشته باشید سه خط زیر را غیرفعال کنید.
\newpage
\thispagestyle{empty}
\vspace*{1.5cm}
\thispagestyle{empty}
\section*{کارنامک}
\vspace*{1cm}
امین روشنی دانش‌آموختۀ دکتری تخصصی (کارشناسی ارشد) رشته آمار از دانشگاه .... در گرایش ..... در سال 1396 است. او در سال 1391 كارشناسی ارشـد (كارشناسـی) خود را از دانشگاه ..... در رشتۀ .... گرایش ..... و كارشناسی (کاردانی) خود را در سال 1387 از دانشگاه ..... در رشتۀ .... دریافت كرد. زمينه‌های پژوهشی او .....، ..... و ..... است.
% برگ تایید هیئت داوران به انگلیسی
\davaranPageEN

% دستوراتی برای نمایه

%{\baselineskip=.6cm
%	\phantomsection
%	\addcontentsline{toc}{chapter}{نمایه}
%	\printindex}

%\include{enRights}
% !TeX root=main.tex
% در این فایل، عنوان پایان‌نامه، مشخصات خود و چکیده پایان‌نامه را به انگلیسی، وارد کنید.

%%%%%%%%%%%%%%%%%%%%%%%%%%%%%%%%%%%%
\baselineskip=.6cm
\begin{latin}
\latinuniversity{Razi University}
\latinfaculty{Science Department}
\latinsubject{Statistics}
\latinfield{Mathematic}
\latintitle{Writing projects, thesis and dissertations using Razi-Thesis Class}
\firstlatinsupervisor{First Supervisor}
\secondlatinsupervisor{Second Supervisor}
\firstlatinadvisor{First Advisor}
\secondlatinadvisor{Second Advisor}
\latinname{Amin}
\latinsurname{Roshani}
\latinthesisdate{January 2022}

\en-abstract{
This thesis studies on writing projects, theses and dissertations using Razi-Thesis Class. It ...
}
\latinkeywords{Writing Thesis, Template, \LaTeX, \XePersian}
\latinfirstPage
\end{latin}

\label{LastPage}

\end{document}